% Options for packages loaded elsewhere
\PassOptionsToPackage{unicode}{hyperref}
\PassOptionsToPackage{hyphens}{url}
\PassOptionsToPackage{dvipsnames,svgnames,x11names}{xcolor}
%
\documentclass[
  letterpaper,
  DIV=11,
  numbers=noendperiod]{scrreprt}

\usepackage{amsmath,amssymb}
\usepackage{iftex}
\ifPDFTeX
  \usepackage[T1]{fontenc}
  \usepackage[utf8]{inputenc}
  \usepackage{textcomp} % provide euro and other symbols
\else % if luatex or xetex
  \usepackage{unicode-math}
  \defaultfontfeatures{Scale=MatchLowercase}
  \defaultfontfeatures[\rmfamily]{Ligatures=TeX,Scale=1}
\fi
\usepackage{lmodern}
\ifPDFTeX\else  
    % xetex/luatex font selection
\fi
% Use upquote if available, for straight quotes in verbatim environments
\IfFileExists{upquote.sty}{\usepackage{upquote}}{}
\IfFileExists{microtype.sty}{% use microtype if available
  \usepackage[]{microtype}
  \UseMicrotypeSet[protrusion]{basicmath} % disable protrusion for tt fonts
}{}
\makeatletter
\@ifundefined{KOMAClassName}{% if non-KOMA class
  \IfFileExists{parskip.sty}{%
    \usepackage{parskip}
  }{% else
    \setlength{\parindent}{0pt}
    \setlength{\parskip}{6pt plus 2pt minus 1pt}}
}{% if KOMA class
  \KOMAoptions{parskip=half}}
\makeatother
\usepackage{xcolor}
\setlength{\emergencystretch}{3em} % prevent overfull lines
\setcounter{secnumdepth}{5}
% Make \paragraph and \subparagraph free-standing
\ifx\paragraph\undefined\else
  \let\oldparagraph\paragraph
  \renewcommand{\paragraph}[1]{\oldparagraph{#1}\mbox{}}
\fi
\ifx\subparagraph\undefined\else
  \let\oldsubparagraph\subparagraph
  \renewcommand{\subparagraph}[1]{\oldsubparagraph{#1}\mbox{}}
\fi


\providecommand{\tightlist}{%
  \setlength{\itemsep}{0pt}\setlength{\parskip}{0pt}}\usepackage{longtable,booktabs,array}
\usepackage{calc} % for calculating minipage widths
% Correct order of tables after \paragraph or \subparagraph
\usepackage{etoolbox}
\makeatletter
\patchcmd\longtable{\par}{\if@noskipsec\mbox{}\fi\par}{}{}
\makeatother
% Allow footnotes in longtable head/foot
\IfFileExists{footnotehyper.sty}{\usepackage{footnotehyper}}{\usepackage{footnote}}
\makesavenoteenv{longtable}
\usepackage{graphicx}
\makeatletter
\def\maxwidth{\ifdim\Gin@nat@width>\linewidth\linewidth\else\Gin@nat@width\fi}
\def\maxheight{\ifdim\Gin@nat@height>\textheight\textheight\else\Gin@nat@height\fi}
\makeatother
% Scale images if necessary, so that they will not overflow the page
% margins by default, and it is still possible to overwrite the defaults
% using explicit options in \includegraphics[width, height, ...]{}
\setkeys{Gin}{width=\maxwidth,height=\maxheight,keepaspectratio}
% Set default figure placement to htbp
\makeatletter
\def\fps@figure{htbp}
\makeatother
\newlength{\cslhangindent}
\setlength{\cslhangindent}{1.5em}
\newlength{\csllabelwidth}
\setlength{\csllabelwidth}{3em}
\newlength{\cslentryspacingunit} % times entry-spacing
\setlength{\cslentryspacingunit}{\parskip}
\newenvironment{CSLReferences}[2] % #1 hanging-ident, #2 entry spacing
 {% don't indent paragraphs
  \setlength{\parindent}{0pt}
  % turn on hanging indent if param 1 is 1
  \ifodd #1
  \let\oldpar\par
  \def\par{\hangindent=\cslhangindent\oldpar}
  \fi
  % set entry spacing
  \setlength{\parskip}{#2\cslentryspacingunit}
 }%
 {}
\usepackage{calc}
\newcommand{\CSLBlock}[1]{#1\hfill\break}
\newcommand{\CSLLeftMargin}[1]{\parbox[t]{\csllabelwidth}{#1}}
\newcommand{\CSLRightInline}[1]{\parbox[t]{\linewidth - \csllabelwidth}{#1}\break}
\newcommand{\CSLIndent}[1]{\hspace{\cslhangindent}#1}

\KOMAoption{captions}{tableheading}
\makeatletter
\makeatother
\makeatletter
\@ifpackageloaded{bookmark}{}{\usepackage{bookmark}}
\makeatother
\makeatletter
\@ifpackageloaded{caption}{}{\usepackage{caption}}
\AtBeginDocument{%
\ifdefined\contentsname
  \renewcommand*\contentsname{Table of contents}
\else
  \newcommand\contentsname{Table of contents}
\fi
\ifdefined\listfigurename
  \renewcommand*\listfigurename{List of Figures}
\else
  \newcommand\listfigurename{List of Figures}
\fi
\ifdefined\listtablename
  \renewcommand*\listtablename{List of Tables}
\else
  \newcommand\listtablename{List of Tables}
\fi
\ifdefined\figurename
  \renewcommand*\figurename{Figure}
\else
  \newcommand\figurename{Figure}
\fi
\ifdefined\tablename
  \renewcommand*\tablename{Table}
\else
  \newcommand\tablename{Table}
\fi
}
\@ifpackageloaded{float}{}{\usepackage{float}}
\floatstyle{ruled}
\@ifundefined{c@chapter}{\newfloat{codelisting}{h}{lop}}{\newfloat{codelisting}{h}{lop}[chapter]}
\floatname{codelisting}{Listing}
\newcommand*\listoflistings{\listof{codelisting}{List of Listings}}
\makeatother
\makeatletter
\@ifpackageloaded{caption}{}{\usepackage{caption}}
\@ifpackageloaded{subcaption}{}{\usepackage{subcaption}}
\makeatother
\makeatletter
\@ifpackageloaded{tcolorbox}{}{\usepackage[skins,breakable]{tcolorbox}}
\makeatother
\makeatletter
\@ifundefined{shadecolor}{\definecolor{shadecolor}{rgb}{.97, .97, .97}}
\makeatother
\makeatletter
\makeatother
\makeatletter
\makeatother
\ifLuaTeX
  \usepackage{selnolig}  % disable illegal ligatures
\fi
\IfFileExists{bookmark.sty}{\usepackage{bookmark}}{\usepackage{hyperref}}
\IfFileExists{xurl.sty}{\usepackage{xurl}}{} % add URL line breaks if available
\urlstyle{same} % disable monospaced font for URLs
\hypersetup{
  pdftitle={Fundamentals of Robotics},
  pdfauthor={Joe(y) Carpinelli},
  colorlinks=true,
  linkcolor={blue},
  filecolor={Maroon},
  citecolor={Blue},
  urlcolor={Blue},
  pdfcreator={LaTeX via pandoc}}

\title{Fundamentals of Robotics}
\usepackage{etoolbox}
\makeatletter
\providecommand{\subtitle}[1]{% add subtitle to \maketitle
  \apptocmd{\@title}{\par {\large #1 \par}}{}{}
}
\makeatother
\subtitle{A review of fundamental concepts in robotics, from coordinate
frames to manipulator dynamics and control.}
\author{Joe(y) Carpinelli}
\date{2023-08-13}

\begin{document}
\maketitle
\ifdefined\Shaded\renewenvironment{Shaded}{\begin{tcolorbox}[enhanced, interior hidden, frame hidden, borderline west={3pt}{0pt}{shadecolor}, boxrule=0pt, sharp corners, breakable]}{\end{tcolorbox}}\fi

\renewcommand*\contentsname{Table of contents}
{
\hypersetup{linkcolor=}
\setcounter{tocdepth}{2}
\tableofcontents
}
\bookmarksetup{startatroot}

\hypertarget{preface}{%
\chapter*{Preface}\label{preface}}
\addcontentsline{toc}{chapter}{Preface}

\markboth{Preface}{Preface}

This is still being written! The following topics will eventually be
explored in this note-set.

\hypertarget{review-of-coordinate-transformations}{%
\subsection*{Review of Coordinate
Transformations}\label{review-of-coordinate-transformations}}
\addcontentsline{toc}{subsection}{Review of Coordinate Transformations}

\begin{itemize}
\tightlist
\item
  Rotations Matrices
\item
  Passive, active rotations
\item
  \(4\times4\) coordinate transformation matrix
\end{itemize}

\hypertarget{kinematic-descriptions}{%
\subsection*{Kinematic Descriptions}\label{kinematic-descriptions}}
\addcontentsline{toc}{subsection}{Kinematic Descriptions}

\begin{itemize}
\tightlist
\item
  Going from robot diagram to modified DH parameters
\item
  Assigning coordinate frames
\end{itemize}

\hypertarget{forward-kinematics}{%
\subsection*{Forward Kinematics}\label{forward-kinematics}}
\addcontentsline{toc}{subsection}{Forward Kinematics}

\begin{itemize}
\tightlist
\item
  Use your kinematic description to map joint angles to Cartesian
  positions \& orientations
\end{itemize}

\hypertarget{translational-rotational-jacobians}{%
\subsection*{Translational \& Rotational
Jacobians}\label{translational-rotational-jacobians}}
\addcontentsline{toc}{subsection}{Translational \& Rotational Jacobians}

\begin{itemize}
\tightlist
\item
  Differentiate your forward kinematic solutions with respect to joint
  angles to find how \textbf{joint velocities} map to \textbf{Cartesian
  velocities} and \textbf{anglular velocities}
\end{itemize}

\hypertarget{inverse-kinematics}{%
\subsection*{Inverse Kinematics}\label{inverse-kinematics}}
\addcontentsline{toc}{subsection}{Inverse Kinematics}

\begin{itemize}
\tightlist
\item
  Find joint angles to match some Cartesian \emph{pose}
\item
  Generally relies on the Jacobian having full rank
\end{itemize}

\hypertarget{singularities}{%
\subsection*{Singularities}\label{singularities}}
\addcontentsline{toc}{subsection}{Singularities}

\begin{itemize}
\tightlist
\item
  Find when the Jacobian does not have full rank
\end{itemize}

\hypertarget{dynamics}{%
\subsection*{Dynamics}\label{dynamics}}
\addcontentsline{toc}{subsection}{Dynamics}

\begin{itemize}
\tightlist
\item
  Apply forces through joint angles
\item
  Incorporate mass properties, and how they interact with Cartesian
  velocities \& angular rates
\item
  Incorporating moments of inertia, other dynamical elements
\end{itemize}

\hypertarget{control}{%
\subsection*{Control}\label{control}}
\addcontentsline{toc}{subsection}{Control}

\begin{itemize}
\tightlist
\item
  PID Control
\end{itemize}

\hypertarget{applications}{%
\subsection*{Applications}\label{applications}}
\addcontentsline{toc}{subsection}{Applications}

\begin{itemize}
\tightlist
\item
  ROS: \emph{Robot Operating System}
\item
  MoveIt: motion planning framework within ROS
\item
  Orocos KDL: Open Source Kinematics \& Dynamics Solvers in C++
\end{itemize}

\bookmarksetup{startatroot}

\hypertarget{reference-frames}{%
\chapter{Reference Frames}\label{reference-frames}}

Reference frames exist at some point in space, at some orientation in
space, relative to another reference frame. There are two ways to
describe multi-axis rotations: fixed axes, and euler (rotating) axes.

\hfill\break

\hypertarget{why-rotations}{%
\section{Why Rotations}\label{why-rotations}}

We use rotations to answer two different questions\ldots{}

\begin{enumerate}
\def\labelenumi{\arabic{enumi}.}
\tightlist
\item
  Given \(P\), what are the coordinates of \textbf{that same point}
  relative to \(F_j\)?

  \begin{itemize}
  \tightlist
  \item
    Answer looks like: \({}^j(P_0) = R_j P\)
  \end{itemize}
\item
  Given \(P\), what are the coordinates of a point \textbf{in} \(F_2\)
  that are rotated about some axis?

  \begin{itemize}
  \tightlist
  \item
    Answer looks like: \({}^2(P) = R_i P\)
  \end{itemize}
\end{enumerate}

How can we reconcile this?

\begin{itemize}
\tightlist
\item
  \(R_j \equiv R_i^T\)
\end{itemize}

\hypertarget{rotation-matrices}{%
\section{Rotation Matrices}\label{rotation-matrices}}

Relating reference frames requires many (somewhat arbitrary)
definitions. We may choose to relate frames by rotating axes, or fixed
axes. We may choose to change the basis of a three-valued vector through
rotation, or we may choose to rotate a three-valued vector in its
original reference frame. For these reasons, you may see two different
definitions of rotation matrices about principle axes. Both definitions
are the transpose --- and due to orthonormality, the inverse --- of one
another.

The definitions below are the principle rotation matrices we will use
moving forward. The following matrices rotate a three-valued vector by
an angle of \(\theta\) about each of the three basis vectors; see
Equations 2.77, 2.78, and 2.79 in Craig's Introduction to Robotics
{[}1{]}.

\begin{equation}\protect\hypertarget{eq-rx}{}{
R_x(\theta) = \begin{bmatrix} 
    1 & 0 & 0 \\
    0 & \cos{\theta} & -\sin{\theta} \\
    0 & \sin{\theta} &  \cos{\theta}
\end{bmatrix}
}\label{eq-rx}\end{equation}

\begin{equation}\protect\hypertarget{eq-ry}{}{
R_y(\theta) = \begin{bmatrix} 
     \cos{\theta} & 0 & \sin{\theta} \\
     0 & 1 & 0 \\
    -\sin{\theta} & 0 & \cos{\theta}
\end{bmatrix}
}\label{eq-ry}\end{equation}

\begin{equation}\protect\hypertarget{eq-rz}{}{
R_z(\theta) = \begin{bmatrix} 
    \cos{\theta} & -\sin{\theta} & 0 \\
    \sin{\theta} &  \cos{\theta} & 0 \\
    0 & 0 & 1 \\
\end{bmatrix}
}\label{eq-rz}\end{equation}

\hypertarget{euler-axis-rotations}{%
\section{Euler Axis Rotations}\label{euler-axis-rotations}}

There are many ways to describe rotations about multiple axes. One way
is to apply a sequence of rotations about principle axes, applying each
rotation to the preceding \emph{intermediate} frame. This technique is
known as \emph{euler} sequence rotations.

You might say: to get frame \(B\), rotate frame \(A\) by \(\theta_1\)
degrees about frame \(A\)'s \(X\) axis, then rotate \textbf{that
intermediate frame} \(\theta_2\) degrees about the intermediate frame's
\(Y\) axis, etc.

Euler ZYX: \(R_z(\theta_3) R_y(\theta_2) R_x(\theta_1)\)

\hypertarget{fixed-axis-rotations}{%
\section{Fixed Axis Rotations}\label{fixed-axis-rotations}}

Fixed XYZ: \(R_x(\theta_1) R_y(\theta_2) R_z(\theta_3)\)

\hypertarget{translations-rotations-transformations}{%
\section{Translations \& Rotations
(Transformations)}\label{translations-rotations-transformations}}

You can combine translations and rotations to create
\emph{transformations}.

\bookmarksetup{startatroot}

\hypertarget{kinematic-descriptions-1}{%
\chapter{Kinematic Descriptions}\label{kinematic-descriptions-1}}

Kinematics is the study of an object's motion independent of the forces
that cause the object's motion. For portability, we want the kinematic
expressions for robotic manipulators to be valid for all manipulators. A
kinematic description, in this context, contains all of the geometric
information about a manipulator necessary to form the standard kinematic
equations of motion. The robotics industry uses several standard
kinematic descriptions, including URDF/XACRO files, and (modified) DH
parameters.

\hfill\break

\hypertarget{modified-dh-parameters}{%
\section{Modified DH Parameters}\label{modified-dh-parameters}}

Given a specific serial manipulator, you may apply the following steps
to construct the Modified DH parameters. Note that Modified DH
Parameters are \textbf{not} unique. You may select multiple sets of
parameters which yield the same kinematics.

\begin{enumerate}
\def\labelenumi{\arabic{enumi}.}
\tightlist
\item
  Copy the contents of Table~\ref{tbl-mdh-template}, where \(N\) is the
  manipulator's number of joints, and \(i\) is the index of the
  \emph{current} joint you're inspecting. The frame with index \(0\)
  should be the \emph{base frame} or \emph{world frame}: some frame from
  which all joint positions can be referenced.
\end{enumerate}

\hypertarget{tbl-mdh-template}{}
\begin{longtable}[]{@{}lllll@{}}
\caption{\label{tbl-mdh-template}Template for Modified DH Parameters
Table}\tabularnewline
\toprule\noalign{}
\(i\) & \(a_{i-1}\) & \(\alpha_{i-1}\) & \(d_i\) & \(\theta_i\) \\
\midrule\noalign{}
\endfirsthead
\toprule\noalign{}
\(i\) & \(a_{i-1}\) & \(\alpha_{i-1}\) & \(d_i\) & \(\theta_i\) \\
\midrule\noalign{}
\endhead
\bottomrule\noalign{}
\endlastfoot
1 & & & & \\
2 & & & & \\
\ldots{} & & & & \\
\(N\) & & & & \\
\end{longtable}

\begin{enumerate}
\def\labelenumi{\arabic{enumi}.}
\setcounter{enumi}{1}
\tightlist
\item
  For each index (row), assign the corresponding link frame axes and
  fill in the values for the parameters \(a\), \(\alpha\), \(d\), and
  \(\theta\) using the rules presented in Figure~\ref{fig-mdh-rules}.
  The index offset \(i-1\) in the column headers can be confusing. To be
  clear, the first values you will fill in are \(a_0\), \(\alpha_0\),
  \(d_1\), and \(\theta_1\). The next values are \(a_1\), \(\alpha_1\),
  \(d_2\), and \(\theta_2\).
\end{enumerate}

\begin{figure}

{\centering 

\begin{enumerate}
\def\labelenumi{\arabic{enumi}.}
\tightlist
\item
  Assign axis \(\hat{z}_i\) to the direction of joint rotation if
  revolute, or joint translation if prismatic. Assign axis \(\hat{x}_i\)
  in any direction that is \textbf{not} the direction of
  \(\hat{z}_{i+1}\); if you can while satisfying this condition,
  assigning \(\hat{x}_i\) in a direction pointing towards the next joint
  is helpful. Place the origin of frame \(i\) where \(\hat{z}_i\) and
  \(\hat{x}_i\) intersect.
\item
  The link length \(a_{i}\) is the distance from \(\hat{z}_i\) to
  \(\hat{z}_{i+1}\) along \(\hat{x}_i\). Usually this value is positive
  or zero; negative \(a_i\) values are discouraged. If your selected
  \(a_i\) is negative, consider changing your definition of
  \(\hat{x}_i\) in Rule \(1\).
\item
  The link twist \(\alpha_{i}\) is the angle from \(\hat{z}_i\) to
  \(\hat{z}_{i+1}\) about \(\hat{x}_i\).
\item
  The joint offset \(d_{i}\) is the distance from \(\hat{x}_{i-1}\) to
  \(\hat{x}_{i}\) along \(\hat{z}_{i}\).
\item
  The joint angle \(\theta_{i}\) is the angle from \(\hat{x}_{i-1}\) to
  \(\hat{x}_{i}\) about \(\hat{z}_{i}\).
\end{enumerate}

}

\caption{\label{fig-mdh-rules}Rules for Selecting Modified DH Parameter
Values}

\end{figure}

\hypertarget{forward-kinematics-1}{%
\section{Forward Kinematics}\label{forward-kinematics-1}}

$$ \begin{equation}
R^i_{i - 1} = \left[
\begin{array}{ccc}
\cos\left( \theta_i \right) &  - \sin\left( \theta_i \right) & 0 \\
\cos\left( \alpha_{i - 1} \right) \sin\left( \theta_i \right) & \cos\left( \alpha_{i - 1} \right) \cos\left( \theta_i \right) &  - \sin\left( \alpha_{i - 1} \right) \\
\sin\left( \alpha_{i - 1} \right) \sin\left( \theta_i \right) & \sin\left( \alpha_{i - 1} \right) \cos\left( \theta_i \right) & \cos\left( \alpha_{i - 1} \right) \\
\end{array}
\right]
\end{equation}
 $$

\hypertarget{eq-mdh-translation}{}
$$ \begin{equation}
P^i_{i - 1} = \left[
\begin{array}{c}
a_{i - 1} \\
 - d_i \sin\left( \alpha_{i - 1} \right) \\
d_i \cos\left( \alpha_{i - 1} \right) \\
\end{array}
\right]
\end{equation}
 $$

$$ \begin{equation}
T^i_{i - 1} = \left[
\begin{array}{cccc}
\cos\left( \theta_i \right) &  - \sin\left( \theta_i \right) & 0 & a_{i - 1} \\
\cos\left( \alpha_{i - 1} \right) \sin\left( \theta_i \right) & \cos\left( \alpha_{i - 1} \right) \cos\left( \theta_i \right) &  - \sin\left( \alpha_{i - 1} \right) &  - d_i \sin\left( \alpha_{i - 1} \right) \\
\sin\left( \alpha_{i - 1} \right) \sin\left( \theta_i \right) & \sin\left( \alpha_{i - 1} \right) \cos\left( \theta_i \right) & \cos\left( \alpha_{i - 1} \right) & d_i \cos\left( \alpha_{i - 1} \right) \\
0 & 0 & 0 & 1 \\
\end{array}
\right]
\end{equation}
 $$

\begin{equation}\protect\hypertarget{eq-transformation-b-to-a}{}{
{}^AP = T_B^A P_B
}\label{eq-transformation-b-to-a}\end{equation} \#\# Examples

Transformations matrices are \textbf{associative}, but are \textbf{not}
commutative.

\begin{equation}\protect\hypertarget{eq-transformation-example}{}{
T_0^4 = T_0^1 T_1^2 T_2^3 T_3^4 
}\label{eq-transformation-example}\end{equation}

\begin{equation}\protect\hypertarget{eq-transformation-inverse}{}{
\text{inv}(T_i^j) \triangleq T_j^i
}\label{eq-transformation-inverse}\end{equation}

We might want to know the end-effector position \(^BP_e\) in the base
frame \(B\).

\[
^BP_e = \text{position}(T_B^0 T_0^1 ... T_N^e)
\]

\bookmarksetup{startatroot}

\hypertarget{jacobians}{%
\chapter{Jacobians}\label{jacobians}}

Jacobians map joint velocities to translational and angular (Cartesian)
velocities. The \emph{angular Cartesian velocities} are commonly
referred to as \emph{body rates}: \(p\) is the angular rate about the
\(X\) axis (roll), \(q\) is the angular rate about the \(Y\) axis
(pitch), and \(r\) is the angular rate about the \(Z\) axis (yaw).

\hfill\break

\hypertarget{derivations}{%
\section{Derivations}\label{derivations}}

The translation and rotation Jacobians can be derived using two methods:
direct differentiation, and the cross-product method.

\begin{equation}\protect\hypertarget{eq-trans-jac}{}{
\begin{bmatrix} \dot{x} \\ \dot{y} \\ \dot{z} \end{bmatrix} = J_T \begin{bmatrix} \dot{q_1} \\ \dot{q_2} \\ ... \\ \dot{q_n} \end{bmatrix}
}\label{eq-trans-jac}\end{equation}

\begin{equation}\protect\hypertarget{eq-rot-jac}{}{
\begin{bmatrix} p \\ q \\ r \end{bmatrix} = J_\Omega \begin{bmatrix} \dot{q_1} \\ \dot{q_2} \\ ... \\ \dot{q_n} \end{bmatrix}
}\label{eq-rot-jac}\end{equation}

\hypertarget{direct-differentiation}{%
\section{Direct Differentiation}\label{direct-differentiation}}

\[ 
I = \left[
\begin{array}{cccc}
\cos\left( \theta_i \right) &  - \sin\left( \theta_i \right) & 0 & a_{i - 1} \\
\cos\left( \alpha_{i - 1} \right) \sin\left( \theta_i \right) & \cos\left( \alpha_{i - 1} \right) \cos\left( \theta_i \right) &  - \sin\left( \alpha_{i - 1} \right) &  - d_i \sin\left( \alpha_{i - 1} \right) \\
\sin\left( \alpha_{i - 1} \right) \sin\left( \theta_i \right) & \sin\left( \alpha_{i - 1} \right) \cos\left( \theta_i \right) & \cos\left( \alpha_{i - 1} \right) & d_i \cos\left( \alpha_{i - 1} \right) \\
0 & 0 & 0 & 1 \\
\end{array}
\right]
\]

\hypertarget{translation-jacobian}{%
\section{Translation Jacobian}\label{translation-jacobian}}

\[
{}^WJ_I = \frac{\partial}{\partial \begin{bmatrix} x& y & z \end{bmatrix}} \begin{bmatrix} a_{i-1} \\ -d_i \sin\alpha_{i-1} \\ d_i \cos\alpha_{i-1} \end{bmatrix}
\]

\bookmarksetup{startatroot}

\hypertarget{references}{%
\chapter*{References}\label{references}}
\addcontentsline{toc}{chapter}{References}

\markboth{References}{References}

\hypertarget{refs}{}
\begin{CSLReferences}{0}{0}
\leavevmode\vadjust pre{\hypertarget{ref-craig}{}}%
\CSLLeftMargin{{[}1{]} }%
\CSLRightInline{Craig, J. J. \emph{Introduction to Robotics}. Pearson
Educacion, 2006.}

\end{CSLReferences}



\end{document}
